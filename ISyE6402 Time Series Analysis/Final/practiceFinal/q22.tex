% Options for packages loaded elsewhere
\PassOptionsToPackage{unicode}{hyperref}
\PassOptionsToPackage{hyphens}{url}
%
\documentclass[
]{article}
\usepackage{lmodern}
\usepackage{amssymb,amsmath}
\usepackage{ifxetex,ifluatex}
\ifnum 0\ifxetex 1\fi\ifluatex 1\fi=0 % if pdftex
  \usepackage[T1]{fontenc}
  \usepackage[utf8]{inputenc}
  \usepackage{textcomp} % provide euro and other symbols
\else % if luatex or xetex
  \usepackage{unicode-math}
  \defaultfontfeatures{Scale=MatchLowercase}
  \defaultfontfeatures[\rmfamily]{Ligatures=TeX,Scale=1}
\fi
% Use upquote if available, for straight quotes in verbatim environments
\IfFileExists{upquote.sty}{\usepackage{upquote}}{}
\IfFileExists{microtype.sty}{% use microtype if available
  \usepackage[]{microtype}
  \UseMicrotypeSet[protrusion]{basicmath} % disable protrusion for tt fonts
}{}
\makeatletter
\@ifundefined{KOMAClassName}{% if non-KOMA class
  \IfFileExists{parskip.sty}{%
    \usepackage{parskip}
  }{% else
    \setlength{\parindent}{0pt}
    \setlength{\parskip}{6pt plus 2pt minus 1pt}}
}{% if KOMA class
  \KOMAoptions{parskip=half}}
\makeatother
\usepackage{xcolor}
\IfFileExists{xurl.sty}{\usepackage{xurl}}{} % add URL line breaks if available
\IfFileExists{bookmark.sty}{\usepackage{bookmark}}{\usepackage{hyperref}}
\hypersetup{
  pdftitle={PracticeFinal},
  pdfauthor={Jim Liu},
  hidelinks,
  pdfcreator={LaTeX via pandoc}}
\urlstyle{same} % disable monospaced font for URLs
\usepackage[margin=1in]{geometry}
\usepackage{color}
\usepackage{fancyvrb}
\newcommand{\VerbBar}{|}
\newcommand{\VERB}{\Verb[commandchars=\\\{\}]}
\DefineVerbatimEnvironment{Highlighting}{Verbatim}{commandchars=\\\{\}}
% Add ',fontsize=\small' for more characters per line
\usepackage{framed}
\definecolor{shadecolor}{RGB}{248,248,248}
\newenvironment{Shaded}{\begin{snugshade}}{\end{snugshade}}
\newcommand{\AlertTok}[1]{\textcolor[rgb]{0.94,0.16,0.16}{#1}}
\newcommand{\AnnotationTok}[1]{\textcolor[rgb]{0.56,0.35,0.01}{\textbf{\textit{#1}}}}
\newcommand{\AttributeTok}[1]{\textcolor[rgb]{0.77,0.63,0.00}{#1}}
\newcommand{\BaseNTok}[1]{\textcolor[rgb]{0.00,0.00,0.81}{#1}}
\newcommand{\BuiltInTok}[1]{#1}
\newcommand{\CharTok}[1]{\textcolor[rgb]{0.31,0.60,0.02}{#1}}
\newcommand{\CommentTok}[1]{\textcolor[rgb]{0.56,0.35,0.01}{\textit{#1}}}
\newcommand{\CommentVarTok}[1]{\textcolor[rgb]{0.56,0.35,0.01}{\textbf{\textit{#1}}}}
\newcommand{\ConstantTok}[1]{\textcolor[rgb]{0.00,0.00,0.00}{#1}}
\newcommand{\ControlFlowTok}[1]{\textcolor[rgb]{0.13,0.29,0.53}{\textbf{#1}}}
\newcommand{\DataTypeTok}[1]{\textcolor[rgb]{0.13,0.29,0.53}{#1}}
\newcommand{\DecValTok}[1]{\textcolor[rgb]{0.00,0.00,0.81}{#1}}
\newcommand{\DocumentationTok}[1]{\textcolor[rgb]{0.56,0.35,0.01}{\textbf{\textit{#1}}}}
\newcommand{\ErrorTok}[1]{\textcolor[rgb]{0.64,0.00,0.00}{\textbf{#1}}}
\newcommand{\ExtensionTok}[1]{#1}
\newcommand{\FloatTok}[1]{\textcolor[rgb]{0.00,0.00,0.81}{#1}}
\newcommand{\FunctionTok}[1]{\textcolor[rgb]{0.00,0.00,0.00}{#1}}
\newcommand{\ImportTok}[1]{#1}
\newcommand{\InformationTok}[1]{\textcolor[rgb]{0.56,0.35,0.01}{\textbf{\textit{#1}}}}
\newcommand{\KeywordTok}[1]{\textcolor[rgb]{0.13,0.29,0.53}{\textbf{#1}}}
\newcommand{\NormalTok}[1]{#1}
\newcommand{\OperatorTok}[1]{\textcolor[rgb]{0.81,0.36,0.00}{\textbf{#1}}}
\newcommand{\OtherTok}[1]{\textcolor[rgb]{0.56,0.35,0.01}{#1}}
\newcommand{\PreprocessorTok}[1]{\textcolor[rgb]{0.56,0.35,0.01}{\textit{#1}}}
\newcommand{\RegionMarkerTok}[1]{#1}
\newcommand{\SpecialCharTok}[1]{\textcolor[rgb]{0.00,0.00,0.00}{#1}}
\newcommand{\SpecialStringTok}[1]{\textcolor[rgb]{0.31,0.60,0.02}{#1}}
\newcommand{\StringTok}[1]{\textcolor[rgb]{0.31,0.60,0.02}{#1}}
\newcommand{\VariableTok}[1]{\textcolor[rgb]{0.00,0.00,0.00}{#1}}
\newcommand{\VerbatimStringTok}[1]{\textcolor[rgb]{0.31,0.60,0.02}{#1}}
\newcommand{\WarningTok}[1]{\textcolor[rgb]{0.56,0.35,0.01}{\textbf{\textit{#1}}}}
\usepackage{graphicx}
\makeatletter
\def\maxwidth{\ifdim\Gin@nat@width>\linewidth\linewidth\else\Gin@nat@width\fi}
\def\maxheight{\ifdim\Gin@nat@height>\textheight\textheight\else\Gin@nat@height\fi}
\makeatother
% Scale images if necessary, so that they will not overflow the page
% margins by default, and it is still possible to overwrite the defaults
% using explicit options in \includegraphics[width, height, ...]{}
\setkeys{Gin}{width=\maxwidth,height=\maxheight,keepaspectratio}
% Set default figure placement to htbp
\makeatletter
\def\fps@figure{htbp}
\makeatother
\setlength{\emergencystretch}{3em} % prevent overfull lines
\providecommand{\tightlist}{%
  \setlength{\itemsep}{0pt}\setlength{\parskip}{0pt}}
\setcounter{secnumdepth}{-\maxdimen} % remove section numbering

\title{PracticeFinal}
\author{Jim Liu}
\date{12/5/2020}

\begin{document}
\maketitle

UK FTSE Stock Index Analysis In this R data analysis, you will examine
daily closing prices for the UK FTSE stock index. You will start by
fitting an ARIMA(0,2,2) model to the data. You may load the data, plot
the data, and fit the ARIMA model

\begin{Shaded}
\begin{Highlighting}[]
\KeywordTok{library}\NormalTok{(TSA)}
\end{Highlighting}
\end{Shaded}

\begin{verbatim}
## 
## Attaching package: 'TSA'
\end{verbatim}

\begin{verbatim}
## The following objects are masked from 'package:stats':
## 
##     acf, arima
\end{verbatim}

\begin{verbatim}
## The following object is masked from 'package:utils':
## 
##     tar
\end{verbatim}

\begin{Shaded}
\begin{Highlighting}[]
\KeywordTok{library}\NormalTok{(mgcv)}
\end{Highlighting}
\end{Shaded}

\begin{verbatim}
## Loading required package: nlme
\end{verbatim}

\begin{verbatim}
## This is mgcv 1.8-33. For overview type 'help("mgcv-package")'.
\end{verbatim}

\begin{Shaded}
\begin{Highlighting}[]
\KeywordTok{library}\NormalTok{(vars)}
\end{Highlighting}
\end{Shaded}

\begin{verbatim}
## Loading required package: MASS
\end{verbatim}

\begin{verbatim}
## Loading required package: strucchange
\end{verbatim}

\begin{verbatim}
## Loading required package: zoo
\end{verbatim}

\begin{verbatim}
## 
## Attaching package: 'zoo'
\end{verbatim}

\begin{verbatim}
## The following objects are masked from 'package:base':
## 
##     as.Date, as.Date.numeric
\end{verbatim}

\begin{verbatim}
## Loading required package: sandwich
\end{verbatim}

\begin{verbatim}
## Loading required package: urca
\end{verbatim}

\begin{verbatim}
## Loading required package: lmtest
\end{verbatim}

\begin{Shaded}
\begin{Highlighting}[]
\KeywordTok{library}\NormalTok{(tseries)}
\end{Highlighting}
\end{Shaded}

\begin{verbatim}
## Registered S3 method overwritten by 'quantmod':
##   method            from
##   as.zoo.data.frame zoo
\end{verbatim}

\begin{Shaded}
\begin{Highlighting}[]
\KeywordTok{library}\NormalTok{(fGarch)}
\end{Highlighting}
\end{Shaded}

\begin{verbatim}
## Loading required package: timeDate
\end{verbatim}

\begin{verbatim}
## 
## Attaching package: 'timeDate'
\end{verbatim}

\begin{verbatim}
## The following objects are masked from 'package:TSA':
## 
##     kurtosis, skewness
\end{verbatim}

\begin{verbatim}
## Loading required package: timeSeries
\end{verbatim}

\begin{verbatim}
## 
## Attaching package: 'timeSeries'
\end{verbatim}

\begin{verbatim}
## The following object is masked from 'package:zoo':
## 
##     time<-
\end{verbatim}

\begin{verbatim}
## Loading required package: fBasics
\end{verbatim}

\begin{Shaded}
\begin{Highlighting}[]
\KeywordTok{library}\NormalTok{(rugarch)}
\end{Highlighting}
\end{Shaded}

\begin{verbatim}
## Loading required package: parallel
\end{verbatim}

\begin{verbatim}
## 
## Attaching package: 'rugarch'
\end{verbatim}

\begin{verbatim}
## The following object is masked from 'package:stats':
## 
##     sigma
\end{verbatim}

\begin{Shaded}
\begin{Highlighting}[]
\KeywordTok{library}\NormalTok{(datasets)}
\KeywordTok{data}\NormalTok{(}\StringTok{"EuStockMarkets"}\NormalTok{)}
\NormalTok{FTSE =}\StringTok{ }\NormalTok{EuStockMarkets[,}\StringTok{"FTSE"}\NormalTok{]}
\KeywordTok{plot}\NormalTok{(FTSE)}
\end{Highlighting}
\end{Shaded}

\includegraphics{q22_files/figure-latex/unnamed-chunk-2-1.pdf}

\begin{Shaded}
\begin{Highlighting}[]
\KeywordTok{plot}\NormalTok{(}\KeywordTok{diff}\NormalTok{(FTSE)); }\KeywordTok{acf}\NormalTok{(}\KeywordTok{diff}\NormalTok{(FTSE)); }\KeywordTok{pacf}\NormalTok{(}\KeywordTok{diff}\NormalTok{(FTSE))}
\end{Highlighting}
\end{Shaded}

\includegraphics{q22_files/figure-latex/unnamed-chunk-2-2.pdf}
\includegraphics{q22_files/figure-latex/unnamed-chunk-2-3.pdf}
\includegraphics{q22_files/figure-latex/unnamed-chunk-2-4.pdf}

\begin{Shaded}
\begin{Highlighting}[]
\KeywordTok{plot}\NormalTok{(}\KeywordTok{diff}\NormalTok{(}\KeywordTok{diff}\NormalTok{(FTSE))); }\KeywordTok{acf}\NormalTok{(}\KeywordTok{diff}\NormalTok{(}\KeywordTok{diff}\NormalTok{(FTSE))); }\KeywordTok{pacf}\NormalTok{(}\KeywordTok{diff}\NormalTok{(}\KeywordTok{diff}\NormalTok{(FTSE)))}
\end{Highlighting}
\end{Shaded}

\includegraphics{q22_files/figure-latex/unnamed-chunk-2-5.pdf}
\includegraphics{q22_files/figure-latex/unnamed-chunk-2-6.pdf}
\includegraphics{q22_files/figure-latex/unnamed-chunk-2-7.pdf}

\begin{Shaded}
\begin{Highlighting}[]
\NormalTok{arima.model =}\StringTok{ }\KeywordTok{arima}\NormalTok{(FTSE, }\KeywordTok{c}\NormalTok{(}\DecValTok{0}\NormalTok{,}\DecValTok{2}\NormalTok{,}\DecValTok{2}\NormalTok{))}
\end{Highlighting}
\end{Shaded}

\begin{enumerate}
\def\labelenumi{\arabic{enumi}.}
\tightlist
\item
  Residual Analysis.
\end{enumerate}

\begin{enumerate}
\def\labelenumi{(\alph{enumi})}
\tightlist
\item
  Let's examine the residuals of the ARIMA(0,2,2) model. Provide the ACF
  and PACF plots for the residuals, as well as a plot of the residuals
  themselves. Is there any sign of heteroskedasticity?
\end{enumerate}

\begin{Shaded}
\begin{Highlighting}[]
\NormalTok{residuals =}\StringTok{ }\KeywordTok{resid}\NormalTok{(arima.model)}
\KeywordTok{plot}\NormalTok{(residuals)}
\end{Highlighting}
\end{Shaded}

\includegraphics{q22_files/figure-latex/unnamed-chunk-3-1.pdf}

\begin{Shaded}
\begin{Highlighting}[]
\KeywordTok{acf}\NormalTok{(residuals, }\DataTypeTok{lag.max =} \DecValTok{20}\NormalTok{)}
\end{Highlighting}
\end{Shaded}

\includegraphics{q22_files/figure-latex/unnamed-chunk-3-2.pdf}

\begin{Shaded}
\begin{Highlighting}[]
\KeywordTok{pacf}\NormalTok{(residuals, }\DataTypeTok{lag.max =} \DecValTok{20}\NormalTok{)}
\end{Highlighting}
\end{Shaded}

\includegraphics{q22_files/figure-latex/unnamed-chunk-3-3.pdf}

\begin{enumerate}
\def\labelenumi{(\alph{enumi})}
\setcounter{enumi}{1}
\tightlist
\item
  Now provide the ACF, PACF, and a plot of the squared residuals. Is
  there any sign of heteroskedasticity?
\end{enumerate}

\begin{Shaded}
\begin{Highlighting}[]
\KeywordTok{plot}\NormalTok{(residuals}\OperatorTok{\^{}}\DecValTok{2}\NormalTok{)}
\end{Highlighting}
\end{Shaded}

\includegraphics{q22_files/figure-latex/unnamed-chunk-4-1.pdf}

\begin{Shaded}
\begin{Highlighting}[]
\KeywordTok{acf}\NormalTok{(residuals}\OperatorTok{\^{}}\DecValTok{2}\NormalTok{, }\DataTypeTok{lag.max =} \DecValTok{20}\NormalTok{)}
\end{Highlighting}
\end{Shaded}

\includegraphics{q22_files/figure-latex/unnamed-chunk-4-2.pdf}

\begin{Shaded}
\begin{Highlighting}[]
\KeywordTok{pacf}\NormalTok{(residuals}\OperatorTok{\^{}}\DecValTok{2}\NormalTok{, }\DataTypeTok{lag.max =} \DecValTok{20}\NormalTok{)}
\end{Highlighting}
\end{Shaded}

\includegraphics{q22_files/figure-latex/unnamed-chunk-4-3.pdf}

\begin{enumerate}
\def\labelenumi{\arabic{enumi}.}
\setcounter{enumi}{1}
\tightlist
\item
  GARCH Modeling.
\end{enumerate}

\begin{enumerate}
\def\labelenumi{(\alph{enumi})}
\tightlist
\item
  Select an ARIMA(0,2,2)-GARCH(m,n) model based on BIC. Consider model
  orders m , n ∈ \{ 0 , 1 , 2 , 3 \}
\end{enumerate}

\begin{Shaded}
\begin{Highlighting}[]
\CommentTok{\#GARCH update}
\NormalTok{test\_modelAGG \textless{}{-}}\StringTok{ }\ControlFlowTok{function}\NormalTok{(m,n)\{}
\NormalTok{  spec =}\StringTok{ }\KeywordTok{ugarchspec}\NormalTok{(}\DataTypeTok{variance.model=}\KeywordTok{list}\NormalTok{(}\DataTypeTok{garchOrder=}\KeywordTok{c}\NormalTok{(m,n)),}
                    \DataTypeTok{mean.model=}\KeywordTok{list}\NormalTok{(}\DataTypeTok{armaOrder=}\KeywordTok{c}\NormalTok{(}\DecValTok{0}\NormalTok{, }\DecValTok{2}\NormalTok{,}\DecValTok{2}\NormalTok{), }
                    \DataTypeTok{include.mean=}\NormalTok{T),}\DataTypeTok{distribution.model=}\StringTok{"std"}\NormalTok{) }

\NormalTok{  fit =}\StringTok{ }\KeywordTok{ugarchfit}\NormalTok{(spec, FTSE, }\DataTypeTok{solver =} \StringTok{\textquotesingle{}hybrid\textquotesingle{}}\NormalTok{)}
\NormalTok{  current.bic =}\StringTok{ }\KeywordTok{infocriteria}\NormalTok{(fit)[}\DecValTok{2}\NormalTok{]}
\NormalTok{  df =}\StringTok{ }\KeywordTok{data.frame}\NormalTok{(m,n,current.bic)}
  \KeywordTok{names}\NormalTok{(df) \textless{}{-}}\StringTok{ }\KeywordTok{c}\NormalTok{(}\StringTok{"m"}\NormalTok{,}\StringTok{"n"}\NormalTok{,}\StringTok{"BIC"}\NormalTok{)}
  \KeywordTok{print}\NormalTok{(}\KeywordTok{paste}\NormalTok{(m,n,current.bic,}\DataTypeTok{sep=}\StringTok{" "}\NormalTok{))}
  \KeywordTok{return}\NormalTok{(df)}
\NormalTok{\}}

\NormalTok{orders =}\StringTok{ }\KeywordTok{data.frame}\NormalTok{(}\OtherTok{Inf}\NormalTok{,}\OtherTok{Inf}\NormalTok{,}\OtherTok{Inf}\NormalTok{)}
\KeywordTok{names}\NormalTok{(orders) \textless{}{-}}\StringTok{ }\KeywordTok{c}\NormalTok{(}\StringTok{"m"}\NormalTok{,}\StringTok{"n"}\NormalTok{,}\StringTok{"BIC"}\NormalTok{)}

\ControlFlowTok{for}\NormalTok{ (m }\ControlFlowTok{in} \DecValTok{0}\OperatorTok{:}\DecValTok{3}\NormalTok{)\{}
     \ControlFlowTok{for}\NormalTok{ (n }\ControlFlowTok{in} \DecValTok{0}\OperatorTok{:}\DecValTok{3}\NormalTok{)\{}
\NormalTok{          possibleError \textless{}{-}}\StringTok{ }\KeywordTok{tryCatch}\NormalTok{(}
\NormalTok{            orders\textless{}{-}}\KeywordTok{rbind}\NormalTok{(orders,}\KeywordTok{test\_modelAGG}\NormalTok{(m,n)),}
            \DataTypeTok{error=}\ControlFlowTok{function}\NormalTok{(e) e}
\NormalTok{          )}
          \ControlFlowTok{if}\NormalTok{(}\KeywordTok{inherits}\NormalTok{(possibleError, }\StringTok{"error"}\NormalTok{)) }\ControlFlowTok{next}
\NormalTok{          \}}
\NormalTok{\}}
\end{Highlighting}
\end{Shaded}

\begin{verbatim}
## [1] "0 0 14.1385450173464"
## [1] "0 1 14.143656313212"
## [1] "0 2 14.1477003339713"
## [1] "0 3 14.1512554648083"
## [1] "1 0 13.2017301581719"
## [1] "1 1 12.9428429492424"
## [1] "1 2 12.9468904528004"
## [1] "1 3 12.9509366237284"
## [1] "2 0 12.9232332367595"
## [1] "2 1 12.8834647008657"
## [1] "2 2 12.8880756652016"
## [1] "2 3 12.8737940590725"
## [1] "3 0 12.8775454827974"
## [1] "3 1 12.881566069064"
## [1] "3 2 12.8856136715675"
## [1] "3 3 12.8773537860348"
\end{verbatim}

\begin{Shaded}
\begin{Highlighting}[]
\NormalTok{orders \textless{}{-}}\StringTok{ }\NormalTok{orders[}\KeywordTok{order}\NormalTok{(}\OperatorTok{{-}}\NormalTok{orders}\OperatorTok{$}\NormalTok{BIC),]}
\KeywordTok{tail}\NormalTok{(orders)}
\end{Highlighting}
\end{Shaded}

\begin{verbatim}
##    m n      BIC
## 16 3 2 12.88561
## 11 2 1 12.88346
## 15 3 1 12.88157
## 14 3 0 12.87755
## 17 3 3 12.87735
## 13 2 3 12.87379
\end{verbatim}

ARIMA(0,2,2)-GARCH(2,3)

\begin{enumerate}
\def\labelenumi{(\alph{enumi})}
\setcounter{enumi}{1}
\tightlist
\item
  Compare the chosen ARIMA-GARCH model to the ARIMA only model (i.e.~the
  ARIMA(0,2,2)-GARCH(0,0) model). Does the GARCH modeling improve the
  model fit?
\end{enumerate}

\begin{Shaded}
\begin{Highlighting}[]
\NormalTok{final.model}\FloatTok{.1}\NormalTok{ =}\StringTok{ }\KeywordTok{garchFit}\NormalTok{(}\OperatorTok{\textasciitilde{}}\StringTok{ }\KeywordTok{arma}\NormalTok{(}\DecValTok{0}\NormalTok{,}\DecValTok{2}\NormalTok{, }\DecValTok{2}\NormalTok{)}\OperatorTok{+}\StringTok{ }\KeywordTok{garch}\NormalTok{(}\DecValTok{2}\NormalTok{,}\DecValTok{3}\NormalTok{), }\DataTypeTok{data=}\NormalTok{FTSE, }\DataTypeTok{trace =} \OtherTok{FALSE}\NormalTok{)}
\end{Highlighting}
\end{Shaded}

\begin{verbatim}
## Warning in sqrt(diag(fit$cvar)): NaNs produced
\end{verbatim}

\begin{verbatim}
## Warning: Using formula(x) is deprecated when x is a character vector of length > 1.
##   Consider formula(paste(x, collapse = " ")) instead.
\end{verbatim}

\begin{Shaded}
\begin{Highlighting}[]
\KeywordTok{summary}\NormalTok{(final.model}\FloatTok{.1}\NormalTok{)}
\end{Highlighting}
\end{Shaded}

\begin{verbatim}
## 
## Title:
##  GARCH Modelling 
## 
## Call:
##  garchFit(formula = ~arma(0, 2, 2) + garch(2, 3), data = FTSE, 
##     trace = FALSE) 
## 
## Mean and Variance Equation:
##  data ~ arma(0, 2, 2) + garch(2, 3)
## <environment: 0x7f95e6018318>
##  [data = FTSE]
## 
## Conditional Distribution:
##  norm 
## 
## Coefficient(s):
##         mu       omega      alpha1      alpha2       beta1       beta2  
## 3.0810e+03  4.7583e+02  9.9280e-01  1.0000e-08  1.0000e-08  1.0000e-08  
##      beta3  
## 1.0000e-08  
## 
## Std. Errors:
##  based on Hessian 
## 
## Error Analysis:
##         Estimate  Std. Error  t value Pr(>|t|)    
## mu     3.081e+03          NA       NA       NA    
## omega  4.758e+02          NA       NA       NA    
## alpha1 9.928e-01   1.169e-01    8.492   <2e-16 ***
## alpha2 1.000e-08   1.247e-01    0.000        1    
## beta1  1.000e-08          NA       NA       NA    
## beta2  1.000e-08          NA       NA       NA    
## beta3  1.000e-08   6.398e-02    0.000        1    
## ---
## Signif. codes:  0 '***' 0.001 '**' 0.01 '*' 0.05 '.' 0.1 ' ' 1
## 
## Log Likelihood:
##  -13831.31    normalized:  -7.436188 
## 
## Description:
##  Sat Dec  5 22:46:17 2020 by user:  
## 
## 
## Standardised Residuals Tests:
##                                 Statistic p-Value     
##  Jarque-Bera Test   R    Chi^2  257.4821  0           
##  Shapiro-Wilk Test  R    W      0.7452978 0           
##  Ljung-Box Test     R    Q(10)  15384.52  0           
##  Ljung-Box Test     R    Q(15)  22212.95  0           
##  Ljung-Box Test     R    Q(20)  28510.9   0           
##  Ljung-Box Test     R^2  Q(10)  35.19167  0.0001158007
##  Ljung-Box Test     R^2  Q(15)  63.00606  7.61765e-08 
##  Ljung-Box Test     R^2  Q(20)  70.49707  1.510612e-07
##  LM Arch Test       R    TR^2   39.54203  8.563849e-05
## 
## Information Criterion Statistics:
##      AIC      BIC      SIC     HQIC 
## 14.87990 14.90071 14.87987 14.88757
\end{verbatim}

\begin{Shaded}
\begin{Highlighting}[]
\NormalTok{model.arima =}\StringTok{ }\KeywordTok{arima}\NormalTok{(FTSE, }\DataTypeTok{order=}\KeywordTok{c}\NormalTok{(}\DecValTok{0}\NormalTok{,}\DecValTok{2}\NormalTok{,}\DecValTok{2}\NormalTok{), }\DataTypeTok{method=}\StringTok{"ML"}\NormalTok{)}
\NormalTok{model.arima}
\end{Highlighting}
\end{Shaded}

\begin{verbatim}
## 
## Call:
## arima(x = FTSE, order = c(0, 2, 2), method = "ML")
## 
## Coefficients:
##           ma1      ma2
##       -0.8720  -0.1280
## s.e.   0.0233   0.0232
## 
## sigma^2 estimated as 923.3:  log likelihood = -8983.22,  aic = 17970.45
\end{verbatim}

\begin{enumerate}
\def\labelenumi{\arabic{enumi}.}
\setcounter{enumi}{2}
\tightlist
\item
  Refine Order Selection.
\end{enumerate}

\begin{enumerate}
\def\labelenumi{(\alph{enumi})}
\item
  Refine the model order selection, i.e.~the choices of p,q,m and n for
  the ARIMA(p,2,q)-GARCH(m,n) model using an appropriate order selection
  process.
\item
  Write out the full mathematical representation of the selected model
  using the parameter estimates.
\end{enumerate}

\begin{enumerate}
\def\labelenumi{\arabic{enumi}.}
\setcounter{enumi}{3}
\tightlist
\item
  Residual Analysis, revisited.
\end{enumerate}

\begin{enumerate}
\def\labelenumi{(\alph{enumi})}
\item
  Plot the residuals and the standardized residuals of the ARIMA-GARCH
  model. How has the GARCH modeling handled the heteroskedasticity?
\item
  Do the standardized residuals of the ARIMA-GARCH model display
  autocorrelation? Use appropriate plots and/or hypothesis tests to
  support your answer.
\item
  Do the squared standardized residuals of the ARIMA-GARCH model display
  autocorrelation? Use appropriate plots and/or hypothesis tests to
  support your answer.
\item
  Do the standardized residuals of the ARIMA-GARCH model follow a normal
  distribution? Use appropriate plots and/or hypothesis tests to support
  your answer.
\end{enumerate}

\begin{enumerate}
\def\labelenumi{\arabic{enumi}.}
\setcounter{enumi}{4}
\tightlist
\item
  Model Fit.
\end{enumerate}

\begin{enumerate}
\def\labelenumi{(\alph{enumi})}
\tightlist
\item
  Use the following code to fit an ARIMA(2,2,2) and an
  ARIMA(2,2,2)-GARCH(1,1) model to the FTSE data.
\end{enumerate}

\begin{Shaded}
\begin{Highlighting}[]
\KeywordTok{library}\NormalTok{(forecast)}
\end{Highlighting}
\end{Shaded}

\begin{verbatim}
## Registered S3 methods overwritten by 'forecast':
##   method       from
##   fitted.Arima TSA 
##   plot.Arima   TSA
\end{verbatim}

\begin{verbatim}
## 
## Attaching package: 'forecast'
\end{verbatim}

\begin{verbatim}
## The following object is masked from 'package:nlme':
## 
##     getResponse
\end{verbatim}

\begin{Shaded}
\begin{Highlighting}[]
\NormalTok{diffs =}\StringTok{ }\KeywordTok{diff}\NormalTok{(}\KeywordTok{diff}\NormalTok{(FTSE))}
\NormalTok{model1 =}\StringTok{ }\KeywordTok{Arima}\NormalTok{(diffs, }\KeywordTok{c}\NormalTok{(}\DecValTok{2}\NormalTok{,}\DecValTok{0}\NormalTok{,}\DecValTok{2}\NormalTok{));}
\NormalTok{model2 =}\StringTok{ }\KeywordTok{garchFit}\NormalTok{(}\OperatorTok{\textasciitilde{}}\StringTok{ }\KeywordTok{arma}\NormalTok{(}\DecValTok{2}\NormalTok{,}\DecValTok{2}\NormalTok{)}\OperatorTok{+}\StringTok{ }\KeywordTok{garch}\NormalTok{(}\DecValTok{1}\NormalTok{,}\DecValTok{1}\NormalTok{), }\DataTypeTok{data =}\NormalTok{ diffs, }\DataTypeTok{trace =} \OtherTok{FALSE}\NormalTok{)}
\end{Highlighting}
\end{Shaded}

\begin{verbatim}
## Warning: Using formula(x) is deprecated when x is a character vector of length > 1.
##   Consider formula(paste(x, collapse = " ")) instead.
\end{verbatim}

Plot the twice differenced data. Based on the plot, which model do you
expect to fit the data better?

\begin{enumerate}
\def\labelenumi{(\alph{enumi})}
\setcounter{enumi}{1}
\tightlist
\item
  Calculate the mean absolute error (MAE) for the two models. Based on
  MAE, which model fits better? How do you explain this result? You may
  use the following commands to get the fitted values:
\end{enumerate}

\begin{Shaded}
\begin{Highlighting}[]
\NormalTok{model1.fitted =}\StringTok{ }\KeywordTok{fitted}\NormalTok{(model1)}
\NormalTok{model2.fitted =}\StringTok{ }\NormalTok{model2}\OperatorTok{@}\NormalTok{fitted}
\end{Highlighting}
\end{Shaded}


\end{document}
